\documentclass[a4paper,9pt]{article}
\usepackage[utf8]{inputenc}
\usepackage[margin=0.2in]{geometry}
\addtolength{\topmargin}{-.6in}



%opening
\title{\textbf{ Design Thinking} \\
        \large Multidisciplinary Project}
\author{Marco Turetta}
\date{2 April 2019}

\begin{document}

\maketitle

\section{Content}
Living in an era of constant turbulence, it is essential to act in the future as well as creating it. We should apply a method that is able to produce platforms for trans-disciplinary co-creation, overtaking the comfort zone through exploring the widest range of possibilities. This method frames problems creatively and generates innovative solutions, it is the so called ‘design thinking’. It is a 'scientific' method that, with the help of 4 basic questions, can lead to innovation. These four questions, what is, what if, what wows, and what works, allow us to generate and experiment with ideas. To understand the specific kinds of problems that design thinking is well suited to, we should see it as an iterative process, human centred and possibility driven. Looking at the design thinking in action, by telling the story of one organization, The Good Kitchen, that addressed the problem of poor nutrition in the elderly through the redesign of a meal delivery service. They used a design thinking approach that allowed them to ask a better question, develop a deeper understanding of stakeholder needs, involve a broad group of stakeholders in co-creating new solutions, and then test these to make sure they were getting them right. Design thinking is about more than just process and tools. Stepping away from the process to look at the role of the designer and his or her mindset, a question rise up: Do I have a mind prepared for innovation? 'Chance favours the prepared mind' (Pasteur) It is a clear statement as simple as powerful, clearly indicating that prepared mind, not luck, is a requirements for innovation. The story of George and Geoff helped us answering that question. A mindset that avoided mistakes and focused on using objective data and analysis had helped George to succeed in a stable environment, but it was leading him towards failure in an unstable one. In contrast to George, the experience of Geoff, whose mindset focused on learning, understanding people as humans, and conducting small experiments instead of doing analysis, prepared him to see and act on opportunities much more successfully than George in an unstable environment. Going back to process and focusing on idea generation as an activity, it was necessary to underline the first two questions: what is, and what if. Another story helped us: Chris Carter, an entrepreneur who saw an area of opportunities he wanted to explore using social networking to help people adopt healthier lifestyles. Chris and the team at Essential Design used interviewing, journaling, and creative projection techniques to get inside the heads of people, helping them to create a set of personas that allowed them to generate a host of creative ideas that met the different needs of the different stakeholders. Moving from idea generation to testing, we analysed the last two questions: what wows, and what works. Through the story of IBM's partnership with GPJ to redesign their trade show experience, we have seen how their front end research into human interaction and learning translated into a series of prototypes that came together in a learning launch conducted at the Sibos trade show in Amsterdam. Eventually, looking at a variety of different organizations, a local municipality, a start-up, and a major corporation, each of which use design thinking to solve very different kinds of problems, we have seen that, even if the outcomes that design thinkers produce may have been unexpected, it was just some disciplined bridge building.

\section{Interpretation}
As a computer engineer I am constantly surrounded by a dynamic and competitive environment. Thus, I think that now more than ever I need to be aware of the creative but structured process that comes before the act of innovation. Engineering is “The creative application of scientific principles to design or develop structures, machines, apparatus…” (Vincenti,1990). I believe that knowing design thinking process will allow me to think outside the box, evaluating options that a rigid approach wouldn’t consider. Traditional engineering mindset has always been guided by the idea of  continuous improvement, aiming for perfection. By contrast, design thinking is geared-up around human imperfection, putting the user into the process. I strongly believe that the most important lesson from design thinking consists in recognizing that the product’s success depends on customer experience as well as engineering.

\section{Applications to our project}
While working on the generation of ideas for our project I have tried to apply design thinking. The first thing I did was to choose the target of my reflection and I thought that tourism could be a field where innovation could be of use, from a humanitarian point of view. I have asked myself the ‘what’ questions: first, I researched information about touristic apps, then I tested them, writing notes about the most used ones. I mixed this works with a research on the tourist’s needs. All of this brought me to generate the idea of an app that possibly could answer all the needs that I listed but were not satisfied by any app. I have tried to put the user before the product. Translating these insights into design criteria was not hard since I was able to collect all of my observations and lay them out using lots of graphic tools (brain storming, projective techniques and mind-mapping). What was fundamental was the feedback that came from my colleagues, and from the users in general (since I was able to gather feedbacks around the internet via surveys): the continuous challenges, coming from the sceptical questions regarding the idea, broadened my mindset. Next step, which consisted in analysing the affordability and feasibility of my ideas, was done by considering the skillset of our group and the economic aspects. This led to the decision of bringing on the concept towards the implementation of a proof of concept.


\end{document}